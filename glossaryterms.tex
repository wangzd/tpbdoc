% \item [3d audio technology] Simulation that creates the illusion of sound sources placed anywhere in 3 dimensional space, including behind, above or below the listener.
% \item[action-event control] Process where a user action creates an physical event.
% \item [API] Application Programming Interface.
% \item[array of microphones] Microphones linked together to expand the effective coverage area. 
% \item [Ausim] 3D audio hardware company.
% \item [Automatic beam steering] Signal processing technique to narrow the microphone coverage area. Used to pick out a speaker and suppress background noise coming from directions other than that of the speaker.
% \item[Benchmarking] A process of researching and observing to understand the state of the art for a given field or topic.
% \item[Brainstorming] A process by which groups of people generate ideas
% \item [Brainwaves] A common term that refers to post-synaptic potentials measured from many neurons in the brain
% \item [CDR] Center for Design Research at Stanford University
% \item [CFP] Otherwise known as a Critical Function Prototype, this is a prototype built to test a concept that is critical to addressing the problem statement.
% \item [Client] Computer program that accesses a server.
% \item [Client-server paradigm] A computing architecture which separates the client from a server over a computer network. 
% \item [Crowded channel] A communication channel that is clogged with information.
% \item [CVE] Acronym for Collaborative Virtual Environment. This is a virtual environment that support more than one user at the same time.
% \item [Dark Horse] An idea that is unlike the others preceding it, an outlier.

\item [远程呈现(telepresence)] 是一种虚拟实在,能够使人实时地以远程的方式于某处出场,即虚拟出场。此时,出场相当于"在场",即你能够在现场之外实时地感知现场,并有效地进行某种操作。

\item [分布式控制系统(distributed control system)] 指控制单元分散于整个系统中,相对独立但互相控制的系统。在本报告中表示多名终端用户相互独立地控制。

\item [HTML5] 是HTML下一个主要的修订版本,现在仍处于发展阶段。目标是取代1999年所制定的HTML 4.01和XHTML 1.0 标准,以期能在互联网应用迅速发展的时候,使网络标准达到符合当代的网络需求。广义论及HTML5时,实际指的是包括HTML、CSS和JavaScript在内的一套技术组合。它希望能够减少浏览器对于需要插件的丰富性网络应用服务(plug-in-based rich internet application,RIA),如Adobe Flash、Microsoft Silverlight,与Oracle JavaFX的需求,并且提供更多能有效增强网络应用的标准集。

\item [WebRTC] 是一个支持网络浏览器进行实时语音对话或视频对话的软件架构。WebRTC 使用GIPS引擎,实现了基于网页的视频会议,并支持722,PCM,ILBC,ISAC等编码,同时使用谷歌的VP8视讯编解码。

\item [WebSocket] 是HTML5开始提供的一种浏览器与服务器间进行全双工通讯的网络技术。在WebSocket API中,浏览器和服务器只需要要做一个握手的动作,然后,浏览器和服务器之间就形成了一条快速通道。两者之间就直接可以数据互相传送。

\item [node.js] 是一个事件驱动I/O服务端JavaScript环境,基于Google的V8引擎。目的是为了提供撰写可扩充网络程序,如Web服务。

\item [mediawiki] 是一套基于网络的Wiki引擎,维基媒体基金会的所有项目乃至众多wiki网站皆采用了这一软件。MediaWiki软件最初是为自由内容百科全书维基百科所开发,今日已被一些公司机构部署为内部的知识管理和内容管理系统。

\item [git] 是一个由林纳斯·托瓦兹为了更好地管理linux内核开发而创立的分布式版本控制/软件配置管理软件。
