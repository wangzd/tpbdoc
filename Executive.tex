%The executive summary is a special chapter before the TOC
% See the other sections (e.g. Context) for more normal chapter headings
%%%%%%%%%%%%%%%%%%%%%%%%%%%%%%%%%%%%
% Call it Front Matter in TOC, as it will include Glossary and auto-generated TOC, LOF.
\chapter[前言]{摘要} 
\label{cha:front}
\addcontentsline{toc}{section}{摘要}
%%%%%%%%%%%%%%%%%%%%%%%%%%%%%%%%%%%%
%Begin the actual executive summary text. If you create any subsections you
%probably want to use  \section*{Section name}  with an asterisk, so they are not numbered.
% Note: to get proper looking quotes use two left/right single quotes: ``. . . ''

%Example of a remark that can be optionally printed:
\begin{remark}\color{blue}
Suggested length of this section: 2 pages including figure(s).
\noindent This the most important section to edit carefully. It should stand alone. Assume it is the \emph{only} section that your corporate liaison's boss will read.
\begin{itemize} \tightlist   % a list with reduced white space
\item Introduce the reader to what your project is about. 
\item Say something brief about the design teams.
\item Motivate the current project direction. The motivation is based on findings from user and expert
interviews, benchmarking, CEP and CFP tests, etc. What interesting findings or insights do you have?
\item \textbf{What you did} is less important than \textbf{what you learned}.
\item Make sure your current ``Point of View'' comes across. The person who reads only the Executive
Summary should still have an idea who your User is.
\item Include one or two images that capture the gist of your design. For Fall, we're probably talking about
pictures of a proposal or vision rather than something you've designed. However, it's possible that something 
from your CFP, CEP or Benchmarking gets the idea across.
\end{itemize}
The remainder of this section is taken from \cite{Autodesk2008Fall}, a pretty good Fall document, done in Latex.
\normalcolor
\end{remark}
%End of remark

%%%%%%%%%%%%%%%%%%BEGIN EXAMPLE TEXT FOR THIS SECTION %%%%%%%%%%%%
%\section*{Example text}

% Engineers must work with distributed teammates around the world. More than ever, designers are tackling all stages of design with remote coworkers whom they may never actually meet face to face. Functioning in this distributed environment can be very challenging both technically and socially. While there are many great tools for managing data and capturing concepts, sharing the output of these tools between distant teammates requires thoughtful planning and continued effort to include distant coworkers during the meeting. Also, distributed team members often feel a sense of isolation - studies have shown that people will collaborate more with people in the same room than with their distributed coworkers who are calling in \cite{Milnethesis}. Developing a way to level the playing field for distributed designers is essential for ever achieving effective distributed design.

在开放便捷的互联网环境下,如何能为希望了解科大校园的求学学子、关心科大
发展的学生家长和科大校友们提供一个方便、真实地了解科大的一个展示平台? 

\begin{figure}[h]
        \centering
                \includegraphics[width=.70\textwidth]{Figures/ch1.school.jpg}
        \caption{校园景观}
        \label{fig:executive}
\end{figure}

就这个需求来讲,某个特定的场所可能有诸多的信息需要开放介绍,甚至需要对来访人员进行一定的导航、指引,而通过单一的静态信息可能无法达到较好的传递效果,以此为出发点,我们力图构建一个系统来改善诸如此类的信息展示功能。

\begin{figure}[h]
        \centering
                \includegraphics[width=.50\textwidth]{Figures/ch1.tele.jpg}
        \caption{静态的远程呈现}
        \label{fig:execimage2}
\end{figure}

对于这种有效信息的获取,其渠道是多元化的,在远程呈现的应用可能下,我们可以将其作为一种获取方式进行设计。让身处远方的“参观者”能够“身临其境”的感受环境,也即通过实时远程传递音频、视频等信息,并且让使用者“自主”的去寻找他们感兴趣的信息。同时,这种功能不只是单向的,通过远程呈现的平台,可以实现两地的人员的实时交流、咨询,也即实现一定的“虚拟出席”的功能,用以辅助环境信息获取的效率。

一个可以令人感兴趣的设计是一种集散控制的导航、参观系统。

旨在某一建筑物中配置一套机群,其移动终端为可移动的并且具有网络传输功能的“远程呈现机器人”,其基本功能是代替参观者虚拟出席到特定环境中,并获取基本的环境信息,对于移动性的控制,一方面可以由非现场的参观者通过某一指标(比如一张平面图的点击)来安全的(操作上存在一定的必要限制)控制移动,另一方面可以由现场人员在移动终端上直接输入指令进行辅助控制,该辅助控制也可以通过前述的平面图点击来封装化的实现。

这一套分布式地远程命令发送与获取,是通过一台服务器的命令转发来实现的,
用户通过网络与服务器建立连接,发送给定的指令,服务器负责将指令集中,再
转发到分散的特定的远程呈现机器人终端,这便是集散式系统的设计。


% A vision for the final product is to better enable dialogue by displaying explicit teammate feedback and participation level visible to the team. Imagine knowing when someone wasn't paying attention, or that your teammate thought you were talking too much, or that everyone really thinks you're idea is pretty cool. All of this could be displayed without saying anything. Aside from simply providing a platform to share information, the system will monitor the quantity of inputs and determine individual participation level, and also offer the opportunity for direct feedback. The objective is to  provide a real-time answer to a common wonder -  what are my teammates really thinking?  

%%%%%%%%%%%%%%%%%%END EXAMPLE TEXT FOR THIS SECTION %%%%%%%


\begin{remark}\color{blue}
\section*{Latex tips:}
\begin{itemize}\tightlist
\item These remarks in blue disappear if you select \textbackslash commentson\{remark\} in me310report.tex
\item Some teams will find the default report cover sheet too plain and will want to change fonts and layout, etc. This is a tedious in Latex. Use Powerpoint or Photoshop or other program to make a nice outer cover which you can pre-pend to the PDF file from Latex.
\item References are linked using the``cite'' command. The template is currently set up to use a bibliography style sheet ``plainurl310.bst'', which is close to the style used by IEEE and other journals with citation numbers in square brackets (e.g., [1]) and printed alphabetically in the Bibliography section. The modifications provide for printing a URL (if there is one) for each reference in a plain format. Numerous other bibliography styles are available, although many do not work with URLs.
\end{itemize}
\normalcolor \end{remark}
