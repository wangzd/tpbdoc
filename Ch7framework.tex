\chapter{整体框架}

整个系统分为三个部分:受控机器人、用户操作端、服务器。

通信建立流程:服务器在80端口上提供服务,受控机器人访问主页,建立视频广
播信道,随后,远程操作者可进入相同的页面来加入广播信道,从而完成p2p的信
道的建立。此后的视频流等的传输不依赖于服务器的控制。

\section{受控机器人}

上位机即为一台笔记本电脑,下位机为机器人移动的驱动部分。上位机连接服务
器,在后台执行机器人的运动控制程序,收取的移动控制指令通过命名管道发送
到运动控制程序,对于运动的实际控制,使用到了机器人自带的超声波传感器,
因此可以避免一些简单的障碍。显示器主界面显示视频通信页面,接收到的多媒
体信息显示在该页面上,即呈现操作用户方面的音频视频信息。

\section{用户操作端}

呈现给用户的是简洁的网页页面,集成了用户所需的多媒体信息,以及相关的控
制操作实现与帮助提示录像。用户通过“注册—登入—加入已存在的信道“,来完成用户需
要的体验。对于机器人的移动操作通过点击操控按钮或者敲击方向键均可实现。

\section{服务器}

提供常开的固定web服务端口,供用户建立连接。设置并管理用户权限,只有通过注册的用户
才可登录,拥有加入音频视频传输以及控制机器人移动的权限。
