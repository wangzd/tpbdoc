\chapter{整体框架}

整个系统分为三个部分:受控机器人、用户操作端、服务器。

通信建立流程:服务器在80端口上提供服务,受控机器人访问主页,建立视频广
播信道,随后,远程操作者可进入相同的页面来加入广播信道,从而完成p2p的信
道的建立。

\section{受控机器人}

上位机即为一台笔记本电脑,下位机为机器人移动的驱动部分。上位机连接服务
器,在后台执行机器人的运动控制程序,收取的移动控制指令通过命名管道发送
到运动控制程序。主界面显示视频通信页面,接收到的多媒体信息显示在该页面
上。

\section{用户操作端}

呈现给用户的是简洁的网页页面,集成了用户所需的多媒体信息,以及相关的控
制操作实现与帮助提示。用户通过注册—登入—加入已存在的信道,来完成用户需
要的体验。对于机器人的移动操作通过点击操控按钮或者敲击方向键均可实现。

\section{服务器}

开放web服务端口,接受用户的连接。设置并管理用户权限,只有通过注册的用户
才可登入,进行之后的操作。
