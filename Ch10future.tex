\chapter{总结与展望}

\section{总结}

当这个项目在2012年10月份起步时,firefox的版本还是16。8个多月过去了,
firefox 22发布在即,这个版本将带来对webRTC的全面支持。

8个月前,没有一款浏览器支持webRTC这个特性,也就是没有一款浏览器能运行
我们的程序,实现远程呈现的功能。8个月过去了,有两款浏览器,firefox与
chrome,在包括Android在内的主流平台上实现了对webRTC的支持,使我们的程
序能够在上亿的设备上运行。

\section{展望}

这两个学期的工作完全集中在软件方面,打造出一个跨平台的通用远程呈现机器
人系统。我希望后续的工作能在硬件方面和软件硬件集成的方面取得进展。

我的设想是,将现有的机器人端控制程序移植到Android平板电脑上,取代笨重
而且待机时间短的笔记本电脑。另一方面,改善机器人的运动系统,使其更加小
型化,外观设计得更好。

当然,我们完成的远程呈现平台也有很大改进的空间,比如更丰富、人性化的控
制机器人的方式。
