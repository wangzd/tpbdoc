\chapter{设计详情及关键实现}

这个系统的特色在于免安装、跨平台、实时性和P2P。

\section{设计特色}

\subsection{免安装}

免安装带来的是用户使用的便捷和免去开发者维护升级的困扰。数据的传输和处
理发生在服务器和浏览器之间,用户只要使用用户名和密码登录,无需下载安装。

另一方面,用户的无需安装也带来了开发者维护升级的方便。我们可以进行小幅
更新而无需发布新的安装包,可以随时修复安全漏洞,也不用同时维护众多版
本——用户使用的版本就是最新的。我们需要做的只是在服务器上部署好,用户需
要做的只是访问我们的主页。

\subsection{跨平台}

随着移动设备狂潮的来临,用户们越来越希望平时享受的服务同样能运行于自己
的移动设备上。另外桌面系统上Windows一家独大的情况不再,Mac用户和Linux
用户的需求日益重要。

借助浏览器这个平台,我们的产品可以实现跨平台化——支持Linux、Android、
Mac OS、Windows等桌面平台和移动平台。可惜的是由于Microsoft和Apple公司
的封闭,目前还无法在iOS和WP平台上使用我们的产品。不过随着时间的推移,
我相信开放的标准必将战胜封闭的平台。

\subsection{实时性}

常见的对Web应用程序的批评是它的相应速度太慢了。确实如此,不过那已经成
为历史了。越来越快的网络带宽和新一代HTML5标准带来的实时性通讯协议
Websocket,带来几乎无延迟的互联网体验。使用我们的系统操纵远程机器人,几
乎感觉不到延迟。

同时新的WebRTC API带来快速、高质量、实时的网络视频的体验。

\subsection{P2P}

没有中央服务器记录您的视频过程,所有的通讯发生于点对点的传输中。这不仅
带来了完全的隐私权和安全性的保证,更带来的是视频会话的快速相应。

\section{关键部分的实现}

运行于服务器端的程序仅仅负责用户的注册、登录,提供静态页面,对控制机器
人的命令的转发。视频通讯发生于用户端的浏览器与机器人端的浏览器之间,完全
没有服务器端的参与。用于控制机器人的程序部署在机器人一方,对用户透明。

\subsection{服务器端}

服务器端使用node.js作为应用服务器。使用express网站框架,采取MVC的设计
模式。使用mongodb存储用户数据。使用socket.io与机器人端进行实时通信。

\subsection{客户端}

网页采用bootstrap进行美化。使用socket.io与服务器进行实时通信,发送对机
器人的控制指令。使用www.freshtilledsoil.com开发的webRTC组件进行实时、
点对点的视频通讯。

\subsection{机器人端}

由于浏览器的沙盒机制,单纯使用网页技术来控制机器人是不切实际的,所以需
要单独的部分来控制机器人。机器人端采用了服务器端相同的node.js进行开发。
为了在任何网络环境中都能够使用,采取主动连接服务器的方式,从而可以在局
域网内正常工作。连接建立后,接受来自服务器转发的来自于用户的控制指令。
随后本部分组件根据接收到的命令控制机器人采取相应动作。
