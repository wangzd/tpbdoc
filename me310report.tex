% % ME310 Report Template -- Started 28 July 2006  Mark Cutkosky
% % Updated: 7Aug08 -LMS; 25Nov10 Mark Cutkosky
% %%%%%%%%%%%%%%%%%%%%%%%%%%%%%%%%%%%

% % Rename this file to whatever you like (e.g. OurFallDoc.tex) and modify the title, authors
% % etc. below. If you rename the section files (e.g. Ch2context.tex) you'll need to change
% % the \include{ } calls in this document as well.

% %%%%%%%%%BEGIN DOCUMENT STYLE SETTINGS%%%%%%%%%%%
% % Don't modify this stuff unless you know what you're doing...
% % We are using the "memoir" class, a widely used set of macros book-like documents.
% % If you get errors that you are missing the "memoir" package you can download and 
% % install it:   http://www.ctan.org/tex-archive/macros/latex/contrib/memoir/

% % memoir document class for standard USA letter paper, printed one side
% \documentclass[11pt,letterpaper,oneside]{memoir}
% \chapterstyle{section}
% \pagestyle{companion}
% \usepackage{graphicx}        % standard LaTeX graphics 
% \usepackage{color}              % support for colored fonts
% \usepackage{url}  \urlstyle{same}     % deal with url strings in bibliography

% \usepackage[pdftex,           %hyperlink cross references, etc.
%     pdfsubject={ME310 Documentation},
%     colorlinks={true},
%     linkcolor={black},
%     citecolor={blue},
%     bookmarksopenlevel=1,
% ]{hyperref}                             

% %The file "me310.sty" should be in the same directory as this file.
% % It contains formatting for page setup, titlepage, glossary, references, etc.
% \usepackage{me310}           
% %%%%%%%END DOCUMENT STYLE SETTINGS%%%%%%%%%%

\documentclass[12pt,letterpaper,oneside]{memoir}

\chapterstyle{section}

\pagestyle{companion}

\usepackage{xeCJK}

\setCJKmainfont{AR PL SungtiL GB}

\xeCJKsetup{CJKglue=\hspace{0pt plus .08 \baselineskip }}

\usepackage{graphicx}

\usepackage{color}

\usepackage{url} \urlstyle{same}

\usepackage{me310}

\usepackage[%pdftex,           %hyperlink cross references, etc.
    xetex,
    pdfsubject={ME310 Documentation},
    colorlinks={true},
    linkcolor={black},
    citecolor={blue},
    bookmarksopenlevel=1,
]{hyperref}                             

\usepackage{setspace}

\linespread{1.4}

\raggedbottom
\usepackage{parskip}
\setlength{\parskip}{5pt}
\setlength{\parindent}{26pt}

%%%%%%%%%%BEGIN TITLE PAGE%%%%%%%%%%%%%%%%

%Replace the strings below with what's right  for you.

%%Insert your Document Title here. Use \\ to force a newline.

%\title{An Interactive Appliance\\
%for Individual Well Being}

\title{基于远程呈现机器人的\\
在线校园参观系统}

%%Enter your team members' names here:      
    \author{王照栋\\
    贾肇聪}

\team{远程呈现机器人组}            % Insert your Team Project Name here.

%% Insert Fall, Winter, Spring here:
\quarter{Fall Design Document}

%% If you don't want it to use the printing date, replace "\today"
%% with the date that you want.
\date{\today}
%%%%%%%%%%%END TITLE PAGE%%%%%%%%%%%%%


%%%%%%%%%BEGIN CUSTOM ABBREVIATIONS%%%%%%%%%
% Define any abbreviations that will apply throughout the document
% to save typing. Examples:
%\def\pmt{{\em Papier M\^{a}ch\`{e}}}  %Define "\pmt" to print "Papier Mache" with accents +1space
%\def\cbike{{\em Casterbike} \,}  %Define "\cbike" to print "Casterbike " italicized +1space
%%%%%%%%%END CUSTOM ABBREVIATIONS%%%%%%%%%

%%%%%%%%DRAFT COMMENTS%%%%%%%%%%%%%%%%
%Allow comments (remarks) to be shown or hidden.
% Put optional text in \begin{remark} ... \end{remark}  environments.
%The usage is a bit counterintuitive: \commentsoff makes them visible; \commentson hides them.
\commentson{remark} %Don't print remarks.
%\commentsoff{remark}  %Do print remarks. 


%%%%%%%%%%%%%%%%%%%%%%%%%%%%%%%%%%%%%%%%
%   BEGIN THE MAIN DOCUMENT
%%%%%%%%%%%%%%%%%%%%%%%%%%%%%%%%%%%%%%%%
\begin{document}

%If you want a figure on the cover page, this is where it goes.
%7 cm is about max figure height before messing up title spacing.
%If making your own fancy coverpage (e.g. in Photoshop) then comment out 
%the \includegraphics{ } and use the \vspace{ } command.
\begin{figure}[t]
% \centering
%   %An example cover image, from 2008 ME310 Kodak project
%   \includegraphics[height= 7cm]{Figures/cover.pdf}
\vspace{3 cm}    %Use this instead if you have no cover picture 
\end{figure}

%%%%%%%%%%%%%%%%%%%%%%%%%%%%%%%%%%%%%%
%Make the title page using arguments defined above.
\titlep

%%%%%%%%%%%%%%%%%%%%%%%%
% Load file "Executive.tex" for the Executive summary.
% Remember, this is a stand-alone section for executives to read.
\chapter[前言]{摘要} 
\label{cha:front}
\addcontentsline{toc}{section}{摘要}

在开放便捷的互联网环境下,如何能为希望了解科大校园的求学学子、关心科大
发展的学生家长和科大校友们提供一个方便、真实地了解科大的一个展示平台? 

\begin{figure}[h]
        \centering
                \includegraphics[width=.70\textwidth]{Figures/ch1.school.jpg}
        \caption{校园景观}
        \label{fig:executive}
\end{figure}

就这个需求来讲,某个特定的场所可能有诸多的信息需要开放介绍,甚至需要对
来访人员进行一定的导航、指引,而通过单一的静态信息可能无法达到较好的传
递效果,以此为出发点,我们力图构建一个系统来改善诸如此类的信息展示功能。

\begin{figure}[h]
        \centering
                \includegraphics[width=.50\textwidth]{Figures/ch1.tele.jpg}
        \caption{静态的远程呈现}
        \label{fig:execimage2}
\end{figure}

对于这种有效信息的获取,其渠道是多元化的,在远程呈现的应用可能下,我们
可以将其作为一种获取方式进行设计。让身处远方的“参观者”能够“身临其境”的
感受环境,也即通过实时远程传递音频、视频等信息,并且让使用者“自主”的去
寻找他们感兴趣的信息。同时,这种功能不只是单向的,通过远程呈现的平台,
可以实现两地的人员的实时交流、咨询,也即实现一定的“虚拟出席”的功能,用
以辅助环境信息获取的效率。

一个可以令人感兴趣的设计是一种集散控制的导航、参观系统。

旨在某一建筑物中配置一套机群,其移动终端为可移动的并且具有网络传输功能
的“远程呈现机器人”,其基本功能是代替参观者虚拟出席到特定环境中,并获取
基本的环境信息,对于移动性的控制,一方面可以由非现场的参观者通过某一指
标(比如一张平面图的点击)来安全的(操作上存在一定的必要限制)控制移动,
另一方面可以由现场人员在移动终端上直接输入指令进行辅助控制,该辅助控制
也可以通过前述的平面图点击来封装化的实现。

这一套分布式地远程命令发送与获取,是通过一台服务器的命令转发来实现的,
用户通过网络与服务器建立连接,发送给定的指令,服务器负责将指令集中,再
转发到分散的特定的远程呈现机器人终端,这便是集散式系统的设计。


%%%%%%%%%%%%%%%%%%%%%%%%
% TOC and LOF are automatically generated
% Make Table of Contents title smaller than a normal Chapter heading:
\renewcommand{\chaptitlefont}{\normalfont\Large\bfseries}
\newpage
\tableofcontents*  %asterisk to prevent it from getting a number

% Optional Lists of Figures and Tables:
\newpage
\listoffigures*  %Note that for this you probably want to add the [short-headings] to captions.
%\listoftables  %I decided to omit the LOT in this example.

%Back to normal size for subsequent sections
\renewcommand{\chaptitlefont}{\normalfont\Huge\bfseries}
%%%%%%%%%%%%%%%%%%%%%%%%

% Set up the Glossary. The template is looking for a file called
% "glossaryterms.tex" with glossary terms and definitions.
% You can either edit this file manually or you
% can use the Memoir glossary feature in which you insert items like
%   \glossary{glossary term}{our definition of what the term means}
% wherever you like, as you write your documentation.
% When you run the report through Latex, it will create a ".glo" file like
% "OurFallDocument.glo" which you can edit to create the file "glossaryterms.tex"
% There is also perl script I made which will do the formatting for you. 
%  perl Glo2Tex OurFallDocument.glo > glossaryterms.tex
\newpage
\section*{缩略词}
\addcontentsline{toc}{section}{缩略词}
\label{sec-glossary}
\begin{description}
% \item [3d audio technology] Simulation that creates the illusion of sound sources placed anywhere in 3 dimensional space, including behind, above or below the listener.
% \item[action-event control] Process where a user action creates an physical event.
% \item [API] Application Programming Interface.
% \item[array of microphones] Microphones linked together to expand the effective coverage area. 
% \item [Ausim] 3D audio hardware company.
% \item [Automatic beam steering] Signal processing technique to narrow the microphone coverage area. Used to pick out a speaker and suppress background noise coming from directions other than that of the speaker.
% \item[Benchmarking] A process of researching and observing to understand the state of the art for a given field or topic.
% \item[Brainstorming] A process by which groups of people generate ideas
% \item [Brainwaves] A common term that refers to post-synaptic potentials measured from many neurons in the brain
% \item [CDR] Center for Design Research at Stanford University
% \item [CFP] Otherwise known as a Critical Function Prototype, this is a prototype built to test a concept that is critical to addressing the problem statement.
% \item [Client] Computer program that accesses a server.
% \item [Client-server paradigm] A computing architecture which separates the client from a server over a computer network. 
% \item [Crowded channel] A communication channel that is clogged with information.
% \item [CVE] Acronym for Collaborative Virtual Environment. This is a virtual environment that support more than one user at the same time.
% \item [Dark Horse] An idea that is unlike the others preceding it, an outlier.

\item [远程呈现(telepresence)] 是一种虚拟实在,能够使人实时地以远程的方式于某处出场,即虚拟出场。此时,出场相当于"在场",即你能够在现场之外实时地感知现场,并有效地进行某种操作。

\item [分布式控制系统(distributed control system)] 指控制单元分散于整个系统中,相对独立但互相控制的系统。在本报告中表示多名终端用户相互独立地控制。

\item [HTML5] 是HTML下一个主要的修订版本,现在仍处于发展阶段。目标是取代1999年所制定的HTML 4.01和XHTML 1.0 标准,以期能在互联网应用迅速发展的时候,使网络标准达到符合当代的网络需求。广义论及HTML5时,实际指的是包括HTML、CSS和JavaScript在内的一套技术组合。它希望能够减少浏览器对于需要插件的丰富性网络应用服务(plug-in-based rich internet application,RIA),如Adobe Flash、Microsoft Silverlight,与Oracle JavaFX的需求,并且提供更多能有效增强网络应用的标准集。

\item [WebRTC] 是一个支持网络浏览器进行实时语音对话或视频对话的软件架构。WebRTC 使用GIPS引擎,实现了基于网页的视频会议,并支持722,PCM,ILBC,ISAC等编码,同时使用谷歌的VP8视讯编解码。

\item [WebSocket] 是HTML5开始提供的一种浏览器与服务器间进行全双工通讯的网络技术。在WebSocket API中,浏览器和服务器只需要要做一个握手的动作,然后,浏览器和服务器之间就形成了一条快速通道。两者之间就直接可以数据互相传送。

\item [node.js] 是一个事件驱动I/O服务端JavaScript环境,基于Google的V8引擎。目的是为了提供撰写可扩充网络程序,如Web服务。

\item [mediawiki] 是一套基于网络的Wiki引擎,维基媒体基金会的所有项目乃至众多wiki网站皆采用了这一软件。MediaWiki软件最初是为自由内容百科全书维基百科所开发,今日已被一些公司机构部署为内部的知识管理和内容管理系统。

\item [git] 是一个由林纳斯·托瓦兹为了更好地管理linux内核开发而创立的分布式版本控制/软件配置管理软件。
   % input the list "glossaryterms.tex"
\end{description}

%%%%%% Example of an optionally printed "remark"
\begin{remark}
\color{blue}
It's a sign of a successful team that the glossary becomes extensive. Define any non-obvious or invented terms. For example, if you reference something by an acronym, that might be a glossary term. Teams also coin terms to describe design features. Define such terms here.  Don't define obvious stuff (axle, keyboard).  

See comments in me310report.tex if you want to generate a glossary semi-automatically from tagged keywords.
\normalcolor
\end{remark}


%%%%%%%%%%%%%%%%%%%%%%%%%%%%%%%%
%% On to the main sections....  Just comment out the \input{} line
%% for any chapters that aren't ready yet.
%Context
\chapter{背景介绍}
\label{sec-context} %Label for cross-referencing

\section{需求陈述}
\label{sec:need}

整体目标:

我们旨在设计出一种实时的远程音视频通信,将两处的信息交流接合,这种设计
基于一台有灵敏移动性能且具备基础的环境障碍感知能力的移动机器人。

背景及意义:

随着计算机、互联网技术的飞速发展,人与人、人与事物之间的联系日益密切,
人们所接触的范围也逐渐广泛起来,于此同时,所需要的信息流量也会大大增加,
传统的传递方式也许并不能很好的起到传递效果。如果让数字化介入其中,便会
收获更好的结果。

试想一下,当某一个机构或部门需要向外界介绍他们的相关信息,这些信息会给
参观者留下非常重要的印象,如果诸如此类的信息能够具有实时性、全方位性,
并且能够充分调动参观者的主观感受,那么这些信息的价值便会大大提升。

通过远程呈现的基本构架,借助移动机器人提供的主观能动性,搭建如此的一个
集散控制的参观导航系统,便会具有如上所述的极佳的效果。

当你身处千里之外,通过简单的互联网界面,点击鼠标、敲击键盘,就可以达到
参观目的地的效果,而且这种信息的获取是实时动态的,该是一件多么惬意的事
情,你一定会对目标地点有一个非常好的主观印象。而且,你还可以随时与那里
的工作人员等互动交流,岂不是更加便捷、实用!

\section{问题陈述}
\label{sec:problem}

进一步分析目标,我们可以将过程中需要着重注意、解决的难题归纳总结,分成
不同的项目部分,以备后续逐步实现预订功能。如下为分类:

\begin{itemize} \tightlist
\item 网络连接搭建的方式

\item 远程操作者的使用界面

\item 机器人上的用户界面

\item 机器人的操控方式

\end{itemize}

对于网络搭建,由于机器人可能部署在任意的网络环境中,因此不能对实现这套
系统的网络有过高的要求与假设。我们的设计目标是部署在机器人上的客户端无
需公网地址段的ip,也无需和使用者处于同一个子网内,只要机器人有一个无线
网络连接,就可以正常工作。

对于使用界面的设定,考虑到跨平台的潜在需求和移动互联网的趋势,应该采用
基于网页界面的设计,具体的功能模块后续的设计中会逐渐添加,进而集成到界
面中,以达到符合用户使用需求的目标。

机器人的控制方式也是一个非常重要的方面,它直接关系到了机器人的安全性等
强制性的因素,而且对于用户体验也是至关重要的。

\section{组员介绍}
\label{sec:team}

\begin{framed}
\noindent \includegraphics[width=50mm]{Figures/context.pic.png}
\parbox[b]{0.6\textwidth}{王照栋\\
现况:中国科学技术大学,自动化系,10级本科生\\
邮件:wangzd@mail.ustc.edu.cn\\
技能:基础机械结构设计,自动控制\\
编程:C语言编程,初步C++面向对象编程,matlab\\
}

来自山东,2010年考入中科大信息学院,后就读于自动化系。喜欢设计,并且具
有一定的动手能力,2012年暑期与同学组队参加过Robogame机器人大赛,并最终
获最佳技术奖单项奖。课余时间比较喜欢参与一些运动以及益智类的活动,热爱
乒乓球、篮球、羽毛球、游泳等运动,对魔方速拧还原有一定的研究。
\end{framed}

\begin{framed}
\noindent
%\parbox[b]{0.6\textwidth}
{贾肇聪\\
现况:中国科学技术大学10级少年班学院\\
}
自由软件爱好者,在本项目中负责linux服务器的管理和网络维护。
\end{framed}
 

%%%%%%%%%%%%%%%%%%%%%%%%%%%%%%%%
%%Design Requirements Chapter
\chapter{设计需求}
\label{sec-requirements}

\begin{remark} \color{blue}
Articulating design requirements is a critical task for a team that starts with a broad problem and needs to determine \emph{what they should design}. After need-finding, and technical and user benchmarking, the team proposes a {\em class of design solutions} that fulfill {\em requirements}\, associated with the problem. In the Fall document, the initial Requirements Definition is the main item of value that teams can deliver to sponsors.

As the design process continues, requirements become more concrete and detailed. New {\em de facto} \, requirements are discovered and documented. Ultimately, competing designs are evaluated with respect to the requirements. If you can't tell whether a design satisfies the requirements, the requirements are too vague.

It is suggested to follow the procedure introduced in Fall quarter lectures and the Paper Bicycle Documents for defining and organizing requirements:
\begin{itemize}\tightlist
\item Requirements (04 Oct. 2011): \url{https://www.stanford.edu/class/archive/me/me310a/me310a.1122/cgi-bin/mediawiki/index.php/FallCalendar} and handout on CloudSafe file server.
\item Paper Bicycle Docs: \url{https://www.stanford.edu/class/archive/me/me310a/me310a.1122/cgi-bin/mediawiki/index.php/PaperBikeDocumentation}
\end{itemize}
\normalcolor
\end{remark}

%This table can  be copied & pasted in your document. The fussy formatting is already set up correctly.
%To get wrapped text, you have use p{} and specify paragraph widths (total < 148mm)
% \begin{table}
% \color{blue}
%   \begin{tabular}{| p{44mm} | p{49mm} | p{42mm} |}   %3 columns, wrapped text
%   \hline
%   Requirement & Metrics & Rationale \\
%   \hline
%   a brief description of what the requirement or objective is & 
%   measurable quantities associated with requirement (how to assess if a design satisfies the requirement) &
%   why this requirement is important or valid \\

% % This example table can be copied and pasted with the text adjusted to meet your needs. &
% % Each column is separated by an ``\&'' sign. The text entries can wrap to more than one line if needed. &
% % Each row is separated by two backslashes and an optional horizontal line. \\
%     \hline
%   \end{tabular}
% \caption[Three column requirements format]{Three column format suggested for requirements (One can make a separate table for each cluster of related requirements).}
%         \label{threecolumnreqs}  %Tag for referring to table
% \normalcolor
% \end{table}

%The remainder of this section contains sample requirements (not an exhaustive set but enough to give an idea) from Autodesk Fall 2007-08 \cite{Autodesk2008Fall} and Audi Fall 2008-09 \cite{Audi2009Fall}.

%%%%%%%%%%%%%%%%%%%%%%%%%%%%%%%%

\section*{简介}

The Autodesk collaboration tool must enhance communication between groups of distributed engineers as they engage in brainstorming.  We have focused on enabling this collaboration via tools that:

\begin{itemize}\tightlist
\item Enable users to communicate naturally and through multiple channels.
\item Enable the team to better utilize their teammates, be they local or distant.
\item Capture the information that was presented.
\end{itemize}

Our benchmarking and prototyping efforts have led to a more detailed definition of what the product needs to be in order to successfully achieve this.  The requirements address what the product functionally needs to do and what it physically needs to be. Because of the wide range of functional opportunities that exist for the product, few physical restrictions are imposed at this stage in the design. 

\section{功能需求}
\label{sec:functionalreqs}

\begin{table}[!h]
        \centering
                \begin{tabular}{| p{42mm} | p{42mm} | p{51mm} |}
                \hline
                \textbf{Requirement}    & \textbf{Metrics} & \textbf{Rationale}\\
                \hline
The product will balance the number of interactions in distributed design meetings among the team members. &    Interactions are questions or statements that develop a concept. The total number of interactions per person during a design meeting will be called $n_{i}$. The solution must reduce the standard deviation of $n_{i}$ between team members as compared to the closest publicly-available competing product.   & The number of times someone interacts in a meeting is one measure of engagement. Brainstorming is a highly social process which thrives on the input from a variety of perspectives. By effectively improving the communication between distributed teams, team members will be more engaged and participate more.\\
\hline
                \end{tabular}
        \caption{Requirement for improved communication}
        \label{tab:mediums1}
\end{table}

\begin{table}[!h]
        \centering
                \begin{tabular}{| p{42mm} | p{42mm} | p{51mm} |}
                \hline
                \textbf{Requirement}    & \textbf{Metrics} & \textbf{Rationale} \\
                \hline
The solution must transmit sound at close to the rate of normal conversation. & The listener must hear the speaker with less than 0.3 seconds lag.      & Audio latency creates a sense of distance. Mobile phone to mobile phone conversations have an average latency of 0.3 seconds, which is noticeable but not disruptive. \\ \hline
Users can capture drawings to share with distributed teammates that are legible. &      Input device must be able to resolve a drawing at 50 points per inch (specifically, they must capture 50 percent contrast modulation at this frequency). &      Drawings by mechanical pencil and ball point pens typically have lines of 0.5mm thickness, which translates to a resolution of 50 points per pinch (ppi).\\ \hline
Users will be able to capture drawings to share with distributed teammates without disrupting the flow of the discussion. & Drawings must be captured and sent within 17 seconds. This is assuming the input device is properly set and there are no external complications. &  We found through benchmarking that sketches are used primarily when describing a concept, and are of little use afterwards. The sketches must be captured and sent before the context of the discussion has changed. Seventeen seconds was found to be about the average comment length during brainstorming in our prototyping. \\ \hline
Users will be able to see the drawings clearly. &       Drawings must be displayed with a resolution of at least 72 ppi.&       The display must be able to resolve at least as a standard computer monitor.\\ 
\hline
                \end{tabular}
        \caption{Required mediums of communication for effective concept development}
        \label{tab:mediums2}
\end{table}

\begin{table}[!h]
        \centering
                \begin{tabular}{| p{42mm} | p{42mm} | p{51mm} |}
                \hline
                \textbf{Requirement}    & \textbf{Metrics} & \textbf{Rationale} \\
                \hline
The tool must be able to be started up quickly for impromptu meetings.  & It must be able to be started in less than 40 seconds. This time is calculated from the moment someone decides to start the system, to the point when the tool is ready to use, with full functionality. If the solution requires use of personal laptops, assume these are already booted up. & Our benchmarking has shown that collaboration tools can fall into disuse if it requires a lengthy setup time. This amount of time is within the range of initiation times for multiple popular conferencing solutions. \\ \hline

        \end{tabular}
        \caption{Social requirements for effective design meetings}
        \label{tab:mediums3}
\end{table}



\newpage

\subsection{功能限制}

\begin{table}[!h]
        \centering
                \begin{tabular}{| p{44mm} | p{49mm} | p{42mm} |}
                \hline
                \textbf{Requirement}    & \textbf{Metrics} & \textbf{Rationale} \\
                \hline
                The bandwidth required must not be prohibitive to standard engineering offices. & The product will require less than 100 Mbps.& The population of potential users would dramatically decrease if the product required more connectivity than a T1 line, which is typically around 100 Mbps.\\ \hline
                        \end{tabular}
        \caption{Functional constraints}
        \label{tab:fconstraints}
\end{table}


\subsection{机遇}

\begin{itemize}\tightlist
\item Utilize existing tools. There are many collaboration and input  tools that exist out there. Our product does not need to be a replacement for them. It could potentially supplement them.
\item Offer new lines of communication:
        \begin{itemize}\tightlist
                \item Facilitate side conversations between distributed users.
                \item Utilize the uncrowded channels offered by other senses than audio/visual, such as tactile.
\end{itemize}
\newpage
\item Be the moderator: 
\begin{itemize}\tightlist
                \item Collect feedback from users directly, via voting, or indirectly. Enable the replacement of video, which conveys very little useful feedback during design meetings.
                \item Encourage the participants to be engaged by monitoring participation.
                \item Display feedback and participation to attendees non-verbally,potentially through the use of avatars.      
\end{itemize}
\item Allow for easier information capture and storage
\begin{itemize}\tightlist
                \item One button information capture
                \item User-driven archiving
\end{itemize}
\item Assist user communication in non-native languages.
        \begin{itemize}\tightlist
                \item Audio buffering
 \end{itemize}
\item The product should be accessible
\begin{itemize}\tightlist
\item Usable for low bandwidth connections for 
\item Be fast to setup.
\item Able to be setup within a typical conference room.
\end{itemize}
\end {itemize}

\subsection{假设}
\begin{itemize}\tightlist
\item Each user has, and is able to use:
\begin{itemize}\tightlist 
\item a personal laptop
\item a mouse
\item a microphone
\end{itemize}
\item Users will speak with a volume of at least 30 dB, as measured when 1 meter from the microphone.
\end{itemize}

%%%%%%PHYSICAL REQUIREMENTS%%%%%%%%%%%
\section{物理需求}
\label{sec:physicalreqs}

\color{blue}
For variety, here is a requirements table from an Audi fall document \cite{Audi2009Fall} done in MS Word and pasted as PDF into Latex. Notice that the fonts are scalable if you zoom in.
\normalcolor

\begin{table}[h]
        \centering
                \includegraphics[width=\textwidth]{Figures/Ch3/Audi08PhysReqs.pdf}
        \caption{Physical Requirements from Audi 2008-09}
        \label{tab:physical-requirements}
\end{table}




%%%%%%%%%%%%%%%%%%%%%%%%%%%%%%%%
% Design Development 
\chapter{设计开发}
\label{sec-development}

\begin{remark}  \color{blue}
\begin{itemize}\tightlist
\item The design is the protagonist of the story; the design team is only a supporting character. 
\item Focus on results (e.g., key findings, insights, lessons learned), not activity (``We brainstormed extensively and eventually settled on two alternative concepts.'')
\item Use lots of images, and not just photographs: diagrams, schematics, flow charts, CAD renderings, etc. are often much more informative than a photo. In any case, use labels pointing to the features you want the reader to appreciate.
\item Lengthy details (e.g. detailed results of technical benchmarking) should go in an Appendix section, with an explicit forward reference from this section.
\item Be professional: for benchmarking, it's essential to properly cite sources of information and provide credits for any images you are using that you did not generate yourselves.
\item Don't refrain from describing ideas that were briefly pursued and dropped. Explain why they were abandoned. In other circumstances they might be worth picking up again.
\item You can use tools such as Pugh concept selection, function-structure diagrams and design decomposition to organize and clarify your design process \cite{Otto07,OttoWood01,UlrichEppinger95}.
\end{itemize}
\normalcolor 
The remaining text in this section contains of excerpts from the Autodesk 2007-08 Fall document \cite{Autodesk2008Fall}.
\end{remark}

首先我们对市场上现有的远程呈现机器人做了一些调查比较。

\begin{figure}[h]
        \begin{center}
                \includegraphics[keepaspectratio, width=4in]{Figures/ch4.our.jpg}
        \caption{AT-3机器人,将来它会变成什么样子?}
        \end{center}
        \label{fig:Design_Development_Flowchart}
\end{figure}

% \section{Brainstorming}
% \label{sec:brainstorm}

%       Our experience in brainstorming was unique in that we were observing and studying our own behavior while exploring solutions. We were constantly studying our own triumphs and shortcomings in the hopes of gaining insight into team dynamics. The results of our multiple brainstorms throughout the fall quarter can be into categorized the following categories:

% \subsection{Communication}

% \begin{figure}[h] 
%       \begin{center}
%               \includegraphics[keepaspectratio, width=3in]{Figures/Ch4/Brainstorm_Communication.jpg}
%       \caption{Key components of communication in design meetings}
%       \label{fig:Brainstorm_Communication}
%       \end{center}
% \end{figure}

% \begin{itemize} \tightlist 
% \item {Open channels}
% \item \begin{itemize} \tightlist Audio and video channels are often inundated with information, even if they are not the most effective means to transmit a piece of information. The team learned that messages are most clearly conveyed when they are free from interference. \end{itemize}
% \item Integrate suggestions quickly
% \item \begin{itemize} \tightlist People can build onto other's ideas immediately, and rapidly change the direction of the conversation. \end{itemize}
% \item \textbf{Verbal communication is the most flexible}
% \item \begin{itemize} \tightlist The team learned from their experience playing cutting-edge multiplayer videogames that verbal communication was the most relied upon medium during fast and slow paced activities. It's versatility and low-bandwidth warranted future attention. \end{itemize}
% \item Gesture
% \item \begin{itemize} \tightlist Gesture is frequently used when explaining an idea. Often, the drawings produced do not look at all like the concept being developed, but the act of drawing in and of itself can be like a gesture, showing how something will work, or where it will be placed, and so forth. \end{itemize}
% \end{itemize}

% \begin{center}
% \color{blue}
% The rest of this subsection is omitted for brevity
% \normalcolor
% \end{center}

% Some key realizations from the brainstorming phase were that social factors and communication shortcomings had alot of opportunity for development. We decided to give special attention to social benchmarking in addition to our technological research.

\section{调查与评估}

市场上已经有将远程呈现机器人技术产业化的例子,但它们的应用似乎都还不广
泛。

\subsection{RP-7i}

\begin{figure}[h] 
\centering
                \includegraphics[keepaspectratio, width=2in]{Figures/ch4.rp7i.jpg}
%Note the use of a short caption tag for the list of figures.
        \caption[RP-7i]{RP-7i医疗服务机器人。价格:50000\$/年\textregistered.}
\end{figure}

我们在IEEE的一片综述中发现了这款机器人。它的使用价格很昂贵,但功能并不
出众,它仅能被医生控制着去检查病人。

\noindent \textbf{关键收获}
\begin{itemize} \tightlist
\item  价格是必须考虑的关键因素。高昂的售价应该是远程机器人不普及的一
  个原因。这也启发了我们将远程呈现机器人作为公共服务提供给广泛的人群免
  费使用。
\end{itemize}

\subsection{TEXAI}

Willow Garage公司是智能机器人领域内的先锋,其推出的TEXAI机器人也是功能
强大。这一点在美剧The Big Bang Theory第四季第二集中得到了充分的展现。

\begin{figure}[h] 
\centering
                \includegraphics[keepaspectratio, width=2in]{Figures/ch4.texai.jpg}        
%Note the use of a short caption tag for the list of figures.
        \caption[TEXAI]{TEXAI \textregistered 是由Willow Garage公司推
          出的一款远程呈现机器人.}
\end{figure}

\noindent \textbf{关键收获}
\begin{itemize} \tightlist
\item  方便的操控性和灵活美观的外观是一款好的设计的必备要素。
\end{itemize}

\section{关键体验原型 (CEP)}

综合之前的需求调查以及基准测试所得到的结果,我们对于远程呈现机器人所
带给使用者的体验原型的进行了初步的确定。其中,通过头脑风暴以及部分试用受访
者的建议,我们总结出了几个比较突出的方面,用以优化CEP。

\subsection{现况}

我们所拥有的基础是一台可以接受上位机指令进行运动动作的移动机器人,为使其能够成为用用户体验使用的远程呈现机器人,需要将指令的发送方从本地抽离,并通过网络传输到达本地,让机器人拥有此项基本功能是进行用户体验测试的基础。

\subsection{起步}

对于搭建关键体验模型的构造与测试基础,我们选取了网页的形式作为操作界面的具体呈现,之所以采用如此形式,是考虑到了网页的诸多优势,比如:网页可以跨平台使用,可移植性较好,页面对于用户的呈现简单,免去了客户端安装等的前缀过程,简洁可靠。

\section{锁定使用者所关注的方面}

通过用户对机器人简单地试用,我们记录了一些条目,用于CEP决策的主要方面。

\subsection{测试与决策}

\begin{enumerate}

\item 控制方式:对于机器人的操作是嵌入网页中的,就其具体实现形式是多样的,比如页面上的按钮点击出发操作指令的下达,键盘上方向键触发操作指令下达。在使用者测试中,当使用者了解到了方向键的控制的可行性后,便会自然地忽略了页面上布置的按钮。

如上结果应该源自移动类物体的方向属性,以及人类本质属性所决定的形象化归类。

\begin{figure}[bht]

  \centering

  \includegraphics[height= 9cm]{Figures/control.pdf}

\end{figure}


\item 稳定性:最原始的测试模型中,远程呈现机器人端的上位机(笔记本电脑)是固定在机器人的最上方,这是最简单的考虑到屏幕位置与人的舒适性所导向的,但是在测试过程中,由于上层支架的高度较高,以及其连接处的稳固性并非十分理想,机器人在移动加速度较大的时候,上位机端的震动会变得十分明显,如果机器人视觉传感也固定在此处的时候,会对画面的品质造成较大的影响。

测试结果,可以更换上位机的目前的状态,比如在支架的高处仅安置用以人机交互的轻质的显示设备,而将上位机重心下移,这一观点的正确定在随后的测试中也得到了一定的验证,将笔记本电脑放置在下部(移动机构附近),即使有较大的移动加速度,上层结构也会有较好的稳定性。

\item 机器人视角:用于测试的视觉传感器为普通的网络摄像头,由于其视角范围较为狭小,无法同时获得近距离路面状况以及前方环境状况的信息,使得操作时给人的感觉并非十分自然。

测试结果,单一简单地摄像头并不能较好的模拟人类的视角,可以在摄像头上附加转动机构,优化视角的范围;另一个解决方案是,使用广角摄像头完成影像的采集。

\item 视觉功能:仅仅完成视觉的传输呈现是最基本的要求,可以在页面上集成一些如图像抓取、视频录制等的附加功能。

\end{enumerate}
        
% \subsubsection{Tactile CFP Concept Development}

% The team wanted to come up with a creative solution that would enhance distance communication. Although we identified software having an important role in our solution, we wanted to try to design something physical. We had to answer these questions that were raised after the benchmarking process:
% \begin{itemize} \tightlist
% \item How can we simulate proximity for remote meetings?
% \item How can we implement action-event control?
% \item What senses can we stimulate that aren't normally used?
% \item What is a low bandwidth solution?
% \end{itemize}

%         The team decided that building a tactile messaging system would solve all four of the aforementioned questions. Tactile messages could replace common interpersonal interaction found in same room meetings. It is normal to welcome each other with a  handshake, make eye contact throughout a meeting, smile at each other, and give high-fives to congratulate others. These occurrences are all absent from distance meetings. A tactile message corresponding to each of these gestures would allow users similar opportunity to communicate as if they were sharing the same physical meeting room.
        
%         The team learned that immersive activities like videogames take advantage of action-event control to offer users a seamless means to interact with their environment. A tactile message could quickly be sent over an open channel and pressing the on button would instantly message the recipient.
        
%         Out of the five senses (sight, hearing, taste, touch, and smell), sight and hearing are the most relied upon during meetings. The team considered possibilities in taste and smell messaging but continued with touch, since delivery of tactile messaging was much more straightforward. Since conventional distance meetings only send and receive auditory and visual information, tactile messages would be distinct and easy to identify. The team believed that tactile messages (high, low, or off) would be low bandwidth.

%         \begin{figure}[h] 
% \centering
%                 \includegraphics[keepaspectratio, width=4in]{Figures/Ch4/tactileCFPschematic.jpg}
%         \caption{The team's whiteboard during a brainstorm session}
% \end{figure}

%         The team wanted to test the effectiveness of tactile messaging and decided against a TCP/IP protocol that sent messages between Stanford and PUJ. The code to write such a protocol was extant and it was unnecessary to include it in our prototype. The team simplified the setup and created two stations separated by physical barriers (a wall and 50' of distance), to simulate a distance meeting. Each station would have a vibrating tactile device for each seated participant at that station and a high/low button assembly to activate the vibrating tactile device for each participant at the other station. Initially the devices were supposed to operate as ''on'' or ''off.'' The team decided that having more variability in the operating speeds of the motors would increase the number of different messages that could be sent, and added a high and low voltage button (1.2V and 0.6V).
%         We were curious to see if effective communication could take place if a distant colleague could see what sketches his distant colleague was drawing. To test this, we used webcams to send live video of what the participants drew on their drawing pads to the other stations.

% \subsubsection{What is critical about this CFP?}
%         The team identified these questions as critical before testing:
% \begin{enumerate} \tightlist
% \item Can it create immersion?
% \item Does it improve upon existing communication tools?
% \item Is it easy to understand?
% \item Is it intuitive?
% \item When should it be used?
% \end{enumerate}

% \begin{figure}[h] 
%         \begin{center}
%                 \includegraphics[width=3in]{Figures/Ch4/tactile_seating.jpg}
%         \caption{The orientation of the two tactile messaging stations.}
%         \end{center}
% \end{figure}

% %A transcript of the meeting with tactile feedback is available in Appendix \ref{sec:tactiletranscript}.

% \subsubsection{Lessons Learned}

%         Tactile sensation is an effective means of communicating contextual information. The messaging system delivered instant vibration between the two stations, helping preserve the flow of conversation without impeding it. Using the vibrations to alert the other users that you wanted to say something was a good way to make comments at the precise time you intended. The tactile devices were \textbf{easy to use} and the participants were encouraged to use them as they saw fit. We noticed that \textbf{vibrations were used most frequently to add emphasis} to accompany laughter, to confirm agreement, offer praise for a good idea and to interrupt the speaker. Interruptions consisted of calls for clarification on a point raised or disagreement with an opinion. Interrupting someone who is speaking can cause the speaker to lose his train of thought or become otherwise agitated. We noticed that \textbf{users preferred to send low speed vibrations} as a gentle interruption as a first attempt to get the speaker's attention. If the first few low speed vibrations did not stop the speaker, the high speed vibrations could be sent, and these usually registered right away. We observed that users reserved high speed vibrations for urgent or important messages. 
        
        
%         The signals were mostly easy to detect, but it was \textbf{not always clear what those signals were trying to communicate}. Ambiguous or superfluous signals distracted the receiving user from the meeting and the confused user would ask, ''Did you just buzz me?'' or ''Why did you buzz me?'' These confused questions would stall the meeting for everyone until the sender was revealed and was able to explain what they were trying to communicate. 
        
%         Vibrations, however, were easily detectable despite loud side conversations, a party in a neighboring room, and constant distractions from people walking by. We attribute this to the fact that the tactile channel is uncrowded compared to the audio channel. In a loud environment it is difficult to pick out audio communication from Skype. Visual distractions make it difficult to focus on the laptop monitor. The tactile sensation rarely stimulated in a teleconference, thus making the slightest vibration very noticeable. 
        
%         We tried two different vibrating interfaces, a vibrating pen and a vibrating wrist patch. The wrist patch was unanimously rejected by the participants because 1) the double stick tape that connected the patch to the user's skin was either too sticky and removed arm hair or not sticky enough after a few uses and would fall off, 2) was tethered to the power supply and restricted movement to the point where the hand with the patch was essentially stationary, 3) vibrations on your wrist are not comfortable, and 4) worry that the patch might give the user an electrical shock. The pen had a practical use, writing, and although the pen was connected to the power supply, the user was not, and the range of motion was adequate enough to write anywhere on the drawing space.
        
%         We finally compared the tactile messaging conference to previous experiences with video conferencing and audio conferencing. These results are summarized in Appendix \ref{sec:Appendix1}.
        
%                 \begin{figure}[h] 
% \centering
%                 \includegraphics[width=0.8\textwidth]{Figures/Ch4/tactile_motor.jpg}
%         \caption[Messaging station wires]{The orientation of the two tactile messaging stations. (Note: the wires connecting the patch to power supply are not in this photo)}
% \end{figure}

%         The tactile messaging critical function prototype was a success in that it definitively answered all the critical questions we asked ourselves before testing.


%%%%%%%%%%%%%%%%%%%%%%%%%%%%%%%%
\chapter{设计描述}
\label{design-description}

\begin{remark}\color{blue}
The description section defines what the design is. If you find yourself adding rationale, or discussing design alternatives, you are writing text that should be moved into the Development section. A few teams find that this section fits more naturally if it comes before the Design Development section.

In the Fall quarter, the design will be in an early stage and this section is largely a proposal for what the design should be (you can call it that, explicitly). Even so, on the basis of preliminary need-finding, benchmarking and critical function evaluation, you have some idea of what is appropriate. Take a point of view and assert it. A CAD model or systems diagram of a concept may be appropriate.
\normalcolor \end{remark}

\section{设计前景}
\label{设计前景}

\begin{remark}\color{blue}
Use this section to describe your vision or proposal for what you think the design might be. Ideally you should have a sketch, a diagram or other images to help define it.
\normalcolor
\end{remark}

%The remaining text in this section contains of excerpts from the Autodesk 2007-08 Fall document \cite{Autodesk2008Fall}.

为用户提供的功能主要分两方面,远程操作者使用功能,现场机器人提供服务功能。

远程操作界面要集成一系列的应用功能。现场机器人要有足够多的自由度、传感器,确保提供服务的全面性、安全性。

\section{机器人CFP}

%The tactile messaging system was comprised of small Jameco vibrating motors (1.3VDC 8,500 RPM) mounted to ball point pens and wrist patches. A simple switchable voltage supply circuit was created to give each vibrating motor a high (1.2V) and low (0.6V) vibrating speed (\ref{fig:tactile_circuit}). Each voltage level was buffered with LM324 opamps, and the circuits were implemented on protoboards. The high and low speeds were selected by switches. 

移动机器人要能够充分实现操作者所需要的行动功能、音视频获取功能的一系列仿人类的功能。

\begin{itemize}
\item 摄像头视角:机器人的视觉是远程呈现的最核心部分,因此作为机器人的眼睛——摄像头的视角自然是需要十分关注的,单单一个具有较大视角的摄像头是不够的,我们需要摄像头有几乎360度的自由度,并且可以被用户操作旋转,以达到期望的视频捕捉效果。
\item 机器人高度:移动底座与机器人“头部”的连接应该采用可以提供直线位移的推杆类结构,确保与现场的人员交互时能够达到比较好的效果。
\item 
机器人的角度:机器人面向的方向不应该由前进方向单一锁定,在高度可以调整的基础上,我们希望能够让机器人的面部能够在一定的控制下有充分的角度可调,甚至可以是全方位的周角。
\item 
交互界面:机器人的交互界面应该是由远程操作者来控制的,比如对话交流的时候,可以显示操作者的面部图像,使得现场人员感觉更加自然一些;在需要现场帮助的时候,例如操作者想去某个位置,但是由于对现场的陌生,无法立刻确认正确的站点,这时可以求助时,在屏幕上显示区位平面图,如果屏幕是可点击的触摸屏,现场人员直接就可以通过点击帮助远处的操作者选定位置。
\item 
传感器:机器人结构外围应该布置有数量足够的传感器,用以保证机器人安全,并且辅助行动。
\end{itemize}

\begin{figure}[h] 
        \begin{center}
                \includegraphics[width= \figwidth]{Figures/ch5.cfp.png}
        \end{center}
        \caption[机器人自由度]{机器人的行动自由度}
        \label{fig:tactile_circuit}  

% \begin{figure}[bhtp] 
%         \begin{center}
%                 \includegraphics[width= \figwidth]{Figures/Ch5/tactile_circuit.jpg}
%         \end{center}
%         \caption[voltage divider]{A simple voltage dividing circuit provided 1.2V (HIGH) and 0.6V (LOW) buffered output voltages for the vibrating motor. Switches triggered the high and low voltages. }
%         \label{fig:tactile_circuit}  
% \end{figure}

% \begin{center}
% \color{blue}
% (Text omitted for brevity)
% \normalcolor
% \end{center}

% Four independent circuits were created to provide messaging to two motors on each side. 90' 16-gauge wire was passed between two stations in the meeting setup shown in \ref{fig:tactile_seating}. Power supplies provided the 9V signal on each side.

% \begin{figure}[bhtp] 
%         \begin{center}
%                 \includegraphics[width=3in]{Figures/Ch5/tactile_seating.jpg}
%         \end{center}
%         \caption[test meeting layout]{Layout of seating during test meeting. Two participants met on one side, with the remote user separated by a wall 50 ft away. }
%         \label{fig:tactile_seating}  
\end{figure}

% In addition to the tactile hardware, Skype was used for video and audio communication. Video was supplied by standard webcams. We mounted the webcams on risers to show video of a sheet of white paper used as the shared drawing space. We chose to focus the video on ideas rather than facial expressions. 


\clearpage
\section{操作网页CFP}

呈献给远程操作者的操作界面是提供远程参观、导航的一个至关重要的接口。该网页界面集合了本系统所有的远程服务功能。

\subsection{数据接口}

%The participation moderator was created by using pre-made desktop software applications called widgets. The desktop was set to a white image, with personal spaces for each participant mapped off by a black boundary and labelled with the participant name. In each personal space, a unique Yahoo! Widgets timer was placed. Unique timer's were used to foster a sense of identity- when glancing at the moderator, the team members could instantly recognize their widget rather than look for their name. 
包括视频传输,音频传输,控制指令采集、传输,反馈信息传输等方面。

视频传输的实时性是主要方面,在网络有延迟的情况下,可以适当牺牲视频品质,
确保流畅、实时的特点。HTML5标准中的WebRTC特性十分符合这个要求。经过简
单测试,发现通过WebRTC,浏览器可以实时地通过浏览器看到摄像头捕获的内容。

音频传输与视频传输要保证一定的同步性。

控制指令的数据量较小,其主要的关注点在于可靠性,保证用户的每个指令都能可靠送达,并且不会由于网络的延迟而误导操作者进行误操作,比如当网络拥堵,用户发送了许多前进的指令,但是机器人接收时有丢包的现象,如果单单是累计确认指令的到达,由于TCP数据包本身的特点,会一并将累积的指令延迟后转入进程,从而使得机器人突然加速前进,造成事故。

另一个方面是控制指令的采集方式,比如敲击键盘,点击按钮,摆动摇杆等操作方式,这些操作依据用户喜好而定,用以增强用户体验。由于直接控制底层的前进、后退等指令对于某些用户过于“危险”,我们可以在界面上提供相应的平面图,并在平片图上设立相应的可点击的“站点”,用户通过点击这些站点,由已经存储的内置路线来操纵机器人相应的移动,并结合传感器来进行相应的避障行走甚至路线修正。

反馈信息是方便用户了解机器人的各项情况,用以辅助操作者进行相应处理。

% \begin{figure}[htbp]
%         \centering
%                 \includegraphics[width=1.00\textwidth]{Figures/Ch5/moderator.pdf}
%         \caption{View of moderator display}
%         \label{fig:moderator}
% \end{figure}

% \begin{center}
% \color{blue}
% (Text omitted for brevity)
% \normalcolor
% \end{center}


% Each was simply a countdown timer with a default starting time, $t_{s}$. As they begin counting down, the amount of time remaining is visible. By clicking twice on any widget, it would reset and begin counting down again from $t_{s}$. The timers were manually reset by one of the teammates during the meeting whenever someone had an interaction. When any timer runs out, it would sound an alarm, designating that the meeting come to a halt until the non-active team member contributes to the conversation. The hypothesis was that, because the timers were visible to the entire team, each member would consciously make an effort to speak before their timer ran out and that no timer would actually buzz, although the rotation of speakers would greatly increase.

% The moderator screen was displayed on a 32" LCD display that was positioned 6' from the center of a table where the group met. The layout is detailed in Figure \ref{fig:moderator_setup}. No video or audio conferencing was used -- all team members were local. The objective of the moderator is to support dialogue in meetings, regardless of whether the members are distributed or not. Audio was recorded of each meeting using Cubase software and an IBM laptop's internal microphone, which was placed in the center of the table so each participant could be heard. 

% \begin{figure}[h]
%         \centering
%                 \includegraphics[width=.75\textwidth]{Figures/Ch5/moderator_setup.jpg}
%         \caption{Layout of design meeting with moderator prototype}
%         \label{fig:moderator_setup}
% \end{figure}

\subsection{权限设置}

% Three meetings were run to test the moderator. The subject of each was the same - our team brainstormed potential final products knowing the key lessons learned after our benchmarking. Three meetings were run in succession, each lasting 30 minutes. The intention of this was to eliminate any personal changes between meetings. For example, if Mike has a really bad day before coming in for a second meeting, he may be much less talkative than in the previous meeting, but not as a result of the moderator. The first meeting served as the control, and no moderator was used. The two subsequent meetings used the moderator with $t_{s}$ at 2 minutes and 1 minute.

% The audio files were analyzed manually by playing back the audio recordings for each meeting and recording the length of each comment that every person made. Fifteen minutes of audio during the middle of each meeting was processed. The data are available in Appendix \ref{sec:Appendix1}. 


不同的用户通过登录来进入使用界面,但是用户是广泛的,我们无法直接管理每个用户的操作,故应该采取一定的权限设置,不同的使用者应该具有不同的使用权限。

比如我们完全信赖的人(我们的工作人员)能够直接操作机器人的各个基本功能,合法的用户不能直接控制机器人的底层移动,但是能够通过部分规定好的路线发出指令是机器人沿线行走,此种服务便会使用到之前陈述的可点击平面图的构想,同时,该类用户能够使用摄像头旋转、俯仰等没有涉及到机器人安全的器件的功能。而陌生的游客便只能通过接受视频等简单地信息来使用远程服务,而不能进行其他的使用操作,如此分级使得机器人的使用变得安全化、高效化。


%%%%%%%%%%%%%%%%%%%%%%%%%%%%%%%%
\chapter{下学期计划}
\label{project-planning}

\begin{remark}\color{blue}
Teams with global partners face special challenges in  terms of organization, project management and planning.
It is a truism that organizational burden goes as the square of team size. 

To address these issues, we ask each local+global team to prepare a \textbf{plan for Winter quarter} to include in this section. You have just accomplished a first, rough critical function prototype (CFP and CEP) and you have given a presentation and written a document that captures the current state of your vision and findings. You have learned who can do what and how much work it really takes. And you are highly motivated to make Winter go more smoothly and to ``take control'' of your project.
\normalcolor
\end{remark}

% \section{计划交付内容}
% Define briefly what is or will be delivered. A short table with some explanatory text could be used here. Your project plan should include the following non-negotiable items and any sub-tasks or intermediate items'' that lead up to them:

% \begin{itemize} \tightlist
% \item Paper Robot (Jan 11-13) -- a mechatronic warm-up for winter
% \item Dark Horse prototype (Jan 25-27) -- a 2nd CFP that probes the edge of the design space
% \item Travel Docs due (Feb 8)
% \item Funky Prototype (Feb 10) -- a CFP where a potential avenue for the final product is developed
% \item Turning Point presentation (Feb 24)
% \item Functional Prototype Review (March 8-10) -- your latest and greatest as Winter quarter draws to a close. It should give a clear indication of what to confidently expect in June.
% \item Winter Design Documents (March 17)
% \end{itemize}

\section{里程碑}
%When are various elements (e.g., rough prototypes, final prototypes) delivered? When are key tests conducted? These are the dates, times, and places where project progress is observable and/or demonstrated. Again, update with planned versus actual dates as the design progresses.

由于采用增量开发模式去开发一个互联网应用,所以里程碑的概念对我们的意义
并不是很大。尽管如此,开发中还是会有几个需要关注的里程碑:
\begin{itemize}
\item 具备基本功能,完全可上线进行测试的原型---预计在四月完成。
\item 重新设计的机器人外形---预计在四月完成。
\item 丰富并完善过的图形界面,后台经历出错后的系统---预计在五月完成。
\item 针对移动互联网客户端兼容的版本---根据前两个里程碑的实现情况决定
  是否选做。
\end{itemize}

% \section{项目时间线}
% Summarize the projected project time line if it is not already explicit in the project planning representations above.

% Use any of the familiar project development representations including lists, Gantt Charts, Pert Charts (Figure \ref{fig:full-page-example}), bubble diagrams, tables, etc. In addition, you will almost certainly need a list or table of items that says a bit
% more about the items and gives an idea who is going to do what.

% \begin{figure}[bhtp] 
% \centering
%                 \includegraphics[width=\textwidth]{Figures/Ch6/su-tmit-after.pdf}
%         \caption[Project task replanning example]{In this example from \cite{Toyota01}, Stanford students collaborated with a group at TMIT, Japan. At the end of the Winter quarter it was decided to abandon one branch of the TMIT effort and to eliminate some of the tight coupling that was originally envisioned. }
%         \label{fig:su-tmit}  %Tag for referring to figure in text.
% \end{figure}

% \begin{figure}[p]   % p for "page" for let it be a full page figure!
% \centering
%                 \includegraphics[angle=90, height= 8in]{Figures/Ch6/ideastorm}
%         \caption[Rotated landscape figure example]{An example of taking a large figure and having Latex rotate it 90 degrees to display it in landscape format as a full page figure.}
%         \label{fig:full-page-example}  %Tag for referring to figure in text.
% \end{figure}

\section{项目管理}
%Explain how your distributed and interdisciplinary team will collaborate, communicate and keep itself on-track with respect to the afore-mentioned deliverables.

到目前为止。在项目管理上我们也采取了去中心的分布式的方法。文档系统被部署到网页可访
问的mediawiki上,文件的共享通过匿名ftp实现。下个学期我们计划进一步实现
分布式的项目管理,包括公开的git服务器,提供源代码版本控制和在线浏览;
公开的bugtracker,接受内部和外部的软件缺陷汇报并用于跟踪开发进度。

\section{预算}
%As with any serious proposal, you should include an estimated budget with some specifics about money that has been spent (Fall) and probably will be spent (Winter). Details on vendors can be put in the Appendix.

剩下的预算还很充分,本学期花了800-900¥,由于放弃了购买昂贵的传感器的
计划,所以下个学期的预算比较充足。

\section{思考和目标}
% This is the one section that you would not find in normal research or engineering proposal. But in the spirit that we're doing this in an academic setting, we want to be sure that we reflect on what we're learning and thinking and where we hope to go with it.

% A part of this may include how your team functioned in the fall - explaining how and why your actual design process deviated from what you originally planned, if relevant. (Time lines and milestones often have the look of having been concocted the night before the report is due.)
在本学期的设计中走了一些弯路,比如需求调查做的不充分,CFP的测试不够充
分等。主要原因还是人手不足和时间不够。所以下个学期一定要做好计划,充分
利用好时间。


\chapter{Dark Horse}

在本学期新开始的dark horse阶段,我们提出了如下两个设计方向供考究。

\begin{itemize}
\item 教学楼内的教学辅助机器人
\item 图书馆里的书籍整理机器人
\end{itemize}

具体的方案如下:

\section{教学楼内的教学辅助机器人}

利用机器人的远程呈现的功能,在多媒体教学设备出现问题时,可以远程的实时
的进行辅助修理,或者远程的实现教学内容的采集与反馈。

\section{图书馆里的书籍整理机器人}

新图书馆采用的电子标签为RFID标签,与此相关,在机器人上加装RFID(如下图)
标签探测,对书架进行循环检测,如果有放错书籍的情况出现,可以远程的反馈
信息。如此来解决许多书本因为放错位置而难以找寻的问题。

\begin{figure}[h]
        \centering
                \includegraphics[width=.70\textwidth]{Figures/ch6.rfid.jpg}
        \caption{RFID}
        \label{fig:rfid}
\end{figure}

对于以上的设计方向,我们在第三教学楼和图书馆中进行了相应的采访、问卷等
方式的具体调查,分析结果如下:

教学设备辅助使用,问题多出现于学期初,且目前突发问题的处理速度并不很慢,
所以实现此用途看来并不合理。教学视频的动态录制,不拘泥于单一的旋转摄像
头的视角,以机器人的灵活视角来捕捉课堂信息,并可通过网络广播。如此一来,
机器人所具备的优势就仅限于自由移动方面,而且考虑到教室的空间自由问题,
这一提议并不十分可行。

图书馆书籍整理,询问了相关图书馆的书籍整理人员,得到的反馈是,这一类问
题是可以被人力来较好的解决的,而且随着技术熟练度的提高,目前已经不是很
大的问题。单纯利用机器人,效率问题难以解决。

最后,在李卫平院长的提议下,我们把最终的目标定位为一个部门、机构专用的
远程虚拟出席设备,利用机器人的灵活移动的特性,能够较好的实现远程现场情
况的采集。

我们的主要任务定位为,将这一功能的实现做的尽量的实用、高效。


\chapter{整体框架}

整个系统分为三个部分:受控机器人、用户操作端、服务器。

通信建立流程:服务器在80端口上提供服务,受控机器人访问主页,建立视频广
播信道,随后,远程操作者可进入相同的页面来加入广播信道,从而完成p2p的信
道的建立。

\section{受控机器人}

上位机即为一台笔记本电脑,下位机为机器人移动的驱动部分。上位机连接服务
器,在后台执行机器人的运动控制程序,收取的移动控制指令通过命名管道发送
到运动控制程序。主界面显示视频通信页面,接收到的多媒体信息显示在该页面
上。

\section{用户操作端}

呈现给用户的是简洁的网页页面,集成了用户所需的多媒体信息,以及相关的控
制操作实现与帮助提示。用户通过注册—登入—加入已存在的信道,来完成用户需
要的体验。对于机器人的移动操作通过点击操控按钮或者敲击方向键均可实现。

\section{服务器}

开放web服务端口,接受用户的连接。设置并管理用户权限,只有通过注册的用户
才可登入,进行之后的操作。


\chapter{设计详情及关键实现}

这个系统的特色在于免安装、跨平台、实时性和P2P。

\section{设计特色}

\subsection{免安装}

免安装带来的使用的便捷和免去维护升级的困扰。数据的传输和处理发生在服务
器和浏览器之间,用户完全可以做到开箱即用。

另一方面,用户的无需安装也带来了开发者维护升级的方便。我们可以进行小幅
更新而无需发布安装包,也不用同时维护众多版本——用户使用的版本就是最新的。
我们需要做的只是在服务器上部署好,用户需要做的只是访问我们的主页。

\subsection{跨平台}

随着移动设备狂潮的来临,用户们越来越希望平时使用的程序同样能运行于自己
的移动设备上。另外桌面系统上Windows一家独大的情况不再,Mac用户和Linux
用户的需求日益重要。

借助浏览器这个平台,我们的产品可以实现跨平台化——支持Linux、Android、
Mac OS、Windows等桌面平台和移动平台。可惜的是由于Microsoft和Apple公司
的封闭,目前还无法在iOS和WP平台上使用我们的产品。不过随着时间的推移,
我相信开放的标准必将战胜封闭的平台。

\subsection{实时性}

常见的对Web应用程序的批评是它的相应速度太慢了。确实如此,不过那已经成
为历史了。越来越快的网络带宽和新一代HTML5标准带来的实时性通讯协议
Websocket,带来几乎无延迟的互联网体验。使用我们的系统操纵远程机器人,几
乎感觉不到延迟。

同时新的WebRTC API带来快速、高质量、实时的网络视频的体验。

\subsection{P2P}

没有中央服务器记录您的视频过程,所有的通讯发生于点对点的传输中。这不仅
带来了完全的隐私权和安全性的保证,更带来的是视频会话的快速相应。

\section{关键部分的实现}

运行于服务器端的程序仅仅负责用户的注册、登录,提供静态页面,对控制机器
人命令的转发。

视频通讯发生于用户端的浏览器与机器人端的浏览器之间,完全没有服务器端的
参与。


\chapter{实现效果}

机器人外形如下

\begin{figure}[h]
        \centering
                \includegraphics[width=.70\textwidth]{Figures/ch9.outlook.jpg}
        \caption{机器人外观}
        \label{fig:outlook}
\end{figure}

用户端主界面如下

\begin{figure}[h]
        \centering
                \includegraphics[width=.70\textwidth]{Figures/ch9.ui.png}
        \caption{用户界面}
        \label{fig:UI}
\end{figure}

这是登入服务页面的效果图,用户通过点击注册来获得有权限的账号。

点击帮助按钮会提供一段视频,展示该服务的使用具体使用方法。

用户登录之后,便进入到相应的有使用权限的页面,如下图:

\begin{figure}[h]
        \centering
                \includegraphics[width=.70\textwidth]{Figures/ch9.user.png}
        \caption{登录后界面}
        \label{fig:user}
\end{figure}

在channel name中输入机器人提供的相关channel信息,点击start即可开始音视频的通讯。
效果如下图,视频由机器人上的广角摄像头拍摄所得。

\begin{figure}[h]
        \centering
                \includegraphics[width=.70\textwidth]{Figures/ch9.view.png}
        \caption{远程呈现}
        \label{fig:view}
\end{figure}

在该页面上,敲击方向键上下左右来控制移动,或者通过直接点击面板上的按钮来实现。


\chapter{总结与展望}

\section{总结}

当这个项目在2012年10月份起步时,firefox的版本还是16。8个多月过去了,
firefox 22发布在即,这个版本将带来对webRTC的全面支持。

8个月前,没有一款浏览器支持webRTC这个特性,也就是没有一款浏览器能运行
我们的程序,实现远程呈现的功能。8个月过去了,有两款浏览器,firefox与
chrome,在包括Android在内的主流平台上实现了对webRTC的支持,使我们的程
序能够在上亿的设备上运行。

\section{展望}

这两个学期的工作完全集中在软件方面,打造出一个跨平台的通用远程呈现机器
人系统。我希望后续的工作能在硬件方面和软件硬件集成的方面取得进展。

我的设想是,将现有的机器人端控制程序移植到Android平板电脑上,取代笨重
而且待机时间短的笔记本电脑。另一方面,改善机器人的运动系统,使其更加小
型化,外观设计得更好。

当然,我们完成的远程呈现平台也有很大改进的空间,比如更丰富、人性化的控
制机器人的方式。


%%%%%%%%%%%%%%%%%%%%%%%%%%%%%%%%
% Appendices are set up same as chapters
\chapter{Reources}
\label{cha:resources}

Include lists of human, institutional and vendor resources here with contact information. This is not for direct citations, which go on the Bibliography.

%%%%%BEGIN BIBLIOGRAPHY%%%%%%%%%
% This is a two-step process in which you first create a ".bib" file, which is processed
% by bibtex into a ".bbl" file for loading into the document. 
% Many online journals and databases now have a feature to automatically download Bib
% citations.  BibDesk is a handy (free) program for Macintosh for managing them.
% EndNote and Refworks (is free to Stanford students) are good alternatives.
\bibliographystyle{plainurl310}     %Modified slightly from plainurl.bst 
\bibliography{me310reportfall}  % Look for file called "me310report.bib" 
%
% Alternatively, you can create your own bibliography list by hand.
% In that case, comment out the last line above and replace with  \input{OurBibFile} 
%%%%%END BIBLIOGRAPHY%%%%%%%%%

%%%%%BEGIN APPENDIX SECTIONS%%%%%%%%%%%%%%%
\appendix
\chapterstyle{default}
 % Appendices are set up same as chapters
% \chapter{Moderator Prototype Data}
% \label{sec:Appendix1}
% Adapted from Autodesk Fall 2007-08 \cite{Autodesk2008Fall}.

% \begin{figure}
%       \centering
%               \includegraphics[width=0.75\textwidth]{Figures/Appendix1/moderatordata.JPG}
%               %Note the use of a short caption tag for the list of figures.
%       \caption[test meetings data]{Length and number of contributions collected from recorded moderator test meetings}
%       \label{fig:moderatordata}
% \end{figure}
\chapter{CC BY-SA 3.0 License}
\subsection*{Attribution-ShareAlike 3.0 Unported  (CC BY-SA 3.0) }
Disclaimer

  The Commons Deed is not a license. It is simply a handy reference for understanding the Legal Code (the full license) — it is a human-readable expression of some of its key terms. Think of it as the user-friendly interface to the Legal Code beneath. This Deed itself has no legal value, and its contents do not appear in the actual license. 


  Creative Commons is not a law firm and does not provide legal services. Distributing of, displaying of, or linking to this Commons Deed does not create an attorney-client relationship. 
 This is a human-readable summary of the Legal Code (the full license).  Disclaimer  This license is acceptable for Free Cultural Works. \subsubsection*{You are free:}
\begin{itemize}
\item \textbf{to Share}
--- to copy, distribute and transmit the work 
\item \textbf{to Remix}
--- to adapt the work 
\item  to make commercial use of the work 

\end{itemize}
\subsubsection*{Under the following conditions:}
\begin{itemize}
\item 

 \textbf{Attribution}
 ---  You must attribute the work in the manner specified by the author or licensor (but not in any way that suggests that they endorse you or your use of the work).   


%  \textbf{ Attribute this work: }
% \\ 
%     Information 
%  What does ``Attribute this work'' mean?  The page you came from contained embedded licensing metadata, including how the creator wishes to be attributed for re-use. You can use the HTML here to cite the work. Doing so will also include metadata on your page so that others can find the original work as well. 
\item 

 \textbf{Share Alike}
 --- If you alter, transform, or build upon this work, you may distribute the resulting work only under the same or similar license to this one.  


\end{itemize}
\subsubsection*{ With the understanding that: }
\begin{itemize}
\item \textbf{Waiver}
 --- Any of the above conditions can be waived if you get permission from the copyright holder. 
\item \textbf{Public Domain}
 --- Where the work or any of its elements is in the public domain under applicable law, that status is in no way affected by the license. 
\item \textbf{Other Rights}
 --- In no way are any of the following rights affected by the license: \begin{itemize}
\item  Your fair dealing or fair use rights, or other applicable copyright exceptions and limitations; 
\item  The author's moral rights; 
\item  Rights other persons may have either in the work itself or in how the work is used, such as publicity or privacy rights. 

\end{itemize}
\end{itemize}
       % There could be multiple appendix files like this
 %%%%%END APPENDIX SECTIONS%%%%%%%%%%%%%%%
 
\end{document}

