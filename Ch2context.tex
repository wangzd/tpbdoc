%Context
\chapter{Context}
\label{sec-context} %Label for cross-referencing

%%%%%%%%%%%%
\begin{remark} \color{blue}
Suggested length: About half a page each for the Need Statement and Problem Statement (plus figures, if any). Another page or two for the design team.
\vspace{0.1in}

\noindent The Context provides background and motivation for your project. It is an enlarged version of the brief context in your Executive Summary.  
\begin{itemize} \tightlist
\item Who came to you with a proposed project area? 
\item What background or context set the stage for your Needfinding and Benchmarking, activities?
\end{itemize}
\normalcolor \end{remark}
%%%%%%%%%%%%

\section{Need Statement}
\label{sec:need}

%%%%%%%%%%%%
\begin{remark} \color{blue}
This section is the high-level result of your user need-finding. It defines the ``Point of View'' or hypothesis that guides your ongoing work. 
\begin{itemize} \tightlist
\item Who wants or needs your product? Why do they want it? Or, what need does the product area address? 
\item What evidence do you have to substantiate the need? Use citations or other evidence you've gathered.
\end{itemize}

\noindent The remaining text is taken from \cite{Autodesk2008Fall}.
\normalcolor \end{remark}
%%%%%%%%%%%%

The design world has changed dramatically in the last decade. The widespread advancement and usage of digital prototyping tools has made it simpler and faster to realize new ideas. At the same time, globalization is requiring designers from remote locations to combine their ideas and make design decisions. 

With the advancement of computational power and communication speed, digital prototyping tools have made it possible to transmit complex drawings around the world. Most digital tools that promote remote collaboration target the idea-to-conception stage of development. The early ideation stages of engineering design, however, are still more effective when discussed locally. The problems of effective communication and effective decision-making in this setting are still largely unsolved. Internet tools setup the virtual meeting space, yet communication is still not as effective as meeting in the same room. Often meeting participants cannot truly work together as they do face-to-face.

Wouldn't it be perfect to have a new tool that focused on the interaction aspects of remote collaboration? A tool that made communication effortless, as if the participants were in the same room. Such a tool could increase the ideation potential of remote meetings and make remote brainstorming a reality. 


\section{Problem Statement}
\label{sec:problem}

%%%%%%%%%%%%%%%
\begin{remark} \color{blue}
Here you get more specific about the particular problem that your design vision is addressing.
The remaining text is again taken from \cite{Autodesk2008Fall},
\normalcolor \end{remark}
%%%%%%%%%%%%%%%

In order to facilitate remote collaboration in the early design stage, we first break down the problem into the following three areas.

\begin{itemize} \tightlist
\item Social Dynamics
\item Communication tools
\item Idea storage and decision making
\end{itemize}

Early ideation is a very social process and requires effective interperson communication. Current teleconferencing tools lack in recreating the level of social dynamics present in face-to-face communication.

Communication tools are a means with which we transmit ideas to each other. This could be either through speaking, body language, or drawing. The early ideation stage requires a rapid exchange of ideas between all participants in a meeting. How can we utilize communication tools effectively to make such a dialog easier?

Finally, the brainstorming stage presents a plethora of ideas that need to be archived and categorized for effective decision making. How can we make it easier for meeting participants to save their ideas and retrieve them later? How can information be viewed to facilitate decision making?

\section{Autodesk}
Since 1982, Autodesk has delivered 2D and 3D visualization tools for clients in manufacturing and design. Some products include the drafting program AutoCAD, digital prototyping software Inventor, and 3D modeler Maya. The company focuses on enhancing the design process by allowing the customer to experience their design through software.

%%%%%%%%%%%%%%%%%%%%%%%%%%%%%%%%%%%%%%%%%%%%%%%%%%

\section{The Design Team}
\label{sec:team}

%%%%%%%%%%%%%%%%%%%
\begin{remark} \color{blue}
See other recent reports for other ways of introducing the team. To the extent that the characteristics of the team influence the project direction, this is of interest to the reader.
\end{remark} \normalcolor
%%%%%%%%%%%%%%%%%%%

Team \pmt, was assembled by the ME310 teaching staff, based on the outcome of Myers-Briggs personality tests (see Table \ref{wildeprefs}) and a desire to create teams with a diversity of interests and backgrounds. There is some evidence that such diversity enhances team creativity \cite{Wilde97} \cite{Wilde07}, even if it creates additional challenges for team management.

% Example table. It gets a caption and reference label.
% The tabular formatting is a bit painful...  An alternative is to use Word
% and insert the PDF printout as for a figure. There are also Word-to-Latex converters.
\begin{table}
  \begin{tabular}{| p{14mm} | p{20mm} | p{20mm} | p{22mm} | p{20mm} | p{12mm} |} 
  \hline
Score & Extroverted-Introverted (E-I) & Intuition-Sensing (N-S) & Feeling-Thinking (F-T) & Perception-Judging (P-J) & Overall \\
People & & & & &\\
\hline
Eric & +6 & +6 & -6 & +12 & ENTP \\
Azin & -18 & +30 & -30 & +18 & INTP \\
Patrick & +6 & +18 & -18 & +6 & ENTP \\
Salil & -18 & +30 & -30 & +18 & INTP \\
\hline
\end{tabular}
\caption{Team preferences scores using the method of Wilde \cite{Wilde07}. }
	\label{wildeprefs}  %Tag for referring to table
\end{table}

\begin{framed}
\noindent \includegraphics[width=50mm]{Figures/Ch2/Scott}
\parbox[b]{0.6\textwidth}{Michael Scott\\
Status: M.E. Graduate Student\\
Contact: mscott@me310.stanford.edu\\
Skills: mechatronics, welding, CNC machining\\
Computing: Solid Works, Matlab, basic C programming, Flash, Dreamweaver\\
}

Born in Paris and raised in New Jersey, I attended Columbia University. For graduate school, I decided to trade in the hustle and bustle of New York City for the sunshine of Palo Alto, and so far I have not been disappointed (though New York will always be dear to my heart). My interests include mechatronics, design (including medical devices), football, tennis, pick-up games, tail-gating, Entourage, South Park.
\end{framed}

\begin{framed}
\noindent \includegraphics[width=50mm]{Figures/Ch2/PamBeesly}
\parbox[b]{0.6\textwidth}{Pamela Beesly\\
Status: M.E. Graduate Student\\
Contact: beesly@me310.stanford.edu\\
Skills: Oxy-acetylene welding, soldering, Lasercamm\\
Computing: Solid Works, Matlab, Java, C++, InDesign, Photoshop\\
}

I grew up in Connecticut, moved to Santa Monica, went to boarding school in New Hampshire. A few things that interest me specifically in design are human-computer interaction, tangible interfaces, usability and how people communicate with computers through interfaces. Outside activities I enjoy are being an editor for a design magazine, social dance, and traveling.
\end{framed}

\begin{framed}
\noindent \includegraphics[width=50mm]{Figures/Ch2/Dwight}
\parbox[b]{0.6\textwidth}{Dwight Schrute\\
Status: M.E. Graduate Student\\
Contact: deputy@me310.stanford.edu\\
Skills: machining, welding, foreign languages\\
Computing: Solid Works, Assembler, Pro/E, Mathematica\\
}

Born in San Diego and raised in Orange County, the ocean has never been far from me. I love all sorts of water activities including bodysurfing, surfing, swimming, and jumping off high objects. I love the outdoors too. I am a sheriff's deputy and expert marksman.\\
\end{framed}

%%%%%%%%%%%%%%%%%%MORE LATEX EXAMPLES%%%%%%%%%%%%%%
% Let's test some in-situ glossary items. The idea is to insert these kinds of definitions anywhere,
% as they occur to you while incrementally working on the documentation.
%See main file for notes about how to convert the ".glo" file to a glossary section.

\glossary {paper weight}{ the combined weight of all paper products (paper, cardboard, crepe paper, paper tape) in the vehicle.}
\glossary {non-paper weight}{ the combined weight of all non paper items (e.g. metal pins, teflon tape, screws). Non-paper weight is severely restricted in ME310 Paper Bike exercises.}
\glossary {papier m\^{a}ch\`{e} }{a composition of shredded paper glued together with library paste}
\glossary {particle board} {a product made of compressed wood chips and resin, often sold in 4 ft x 8 ft sheets. Particle board is not considered a paper product.}
\glossary {sound deadener board} {a product made of compressed cardboard fibers and resin, often sold in 4 ft x 8 ft sheets. Sound deadener board is not considered a paper product.}

\begin{remark} \color{blue}
\section*{Experiments with tables in Latex}

Here is a centered tabular form that is 3/5 of the current text width and has a horizontal line but no
vertical lines:

 \begin{center}  % put inside center environment
  \begin{tabular*}{0.6 \textwidth}%
     {@{\extracolsep{\fill}}cccr}
  \multicolumn{3}{c}{a label spanning 3 merged cells} & label 4  \\
  \hline  % put a line under headers
  item 1a  & item 2a  & item 3a  & item 4a  \\
  item 1b  & item 2c  & item 3c  & item 4c  \\
  \end{tabular*}
  \end{center}

\noindent For anything more complicated than the examples in this section, it may be easiest to do the table in MS Word or OpenOffice, create a pdf and include the pdf in a table environment, as done in Table \ref{tab:physical-requirements}. Because pdf files have scalable fonts, the print resolution will be good.

\end{remark}\normalcolor
